\documentclass[11pt, letterpaper]{awesome-cv}
\usepackage{paracol}
\usepackage{progressbar}
\usepackage{tagging}

% Tags are inserted here by pdf.py:
%%% TAGS %%%

% When editing, manually enable sections:
\iftrue
\usetag{technical-skills}
\usetag{languages-and-frameworks}
\usetag{tools}
\usetag{experience}
\usetag{mercury}
\usetag{microsemi}
\usetag{arxan-defense-systems}
% \usetag{arxan}
% \usetag{rose-hulman}
% \usetag{ventures}
% \usetag{perry-schools}
% \usetag{perritech}
\usetag{education}
\usetag{masters}
\usetag{bachelors}
% \usetag{high-school}
\usetag{projects}
\usetag{serial-bridge}
\usetag{chalk}
\usetag{point-vote}
\usetag{ninja}
\usetag{sprint}
\usetag{noisebot}
% \usetag{spades}
% \usetag{got}
% \usetag{lync-helper}
% \usetag{gir}
% \usetag{woop}
% \usetag{publications}
% \usetag{subverting-the-fundamentals-sequence}
\fi

% Settings
\geometry{left=1cm, top=.8cm, right=1cm, bottom=1cm, footskip=.5cm}
\colorlet{awesome}{awesome-nephritis}
\renewcommand{\acvSectionTopSkip}{1mm}
\progressbarchange{ticksheight=0}

% Personal info
\name{Michael}{Mrozek}
\photo[edge,right]{photo}
\position{Software Engineer}
\email{resume@mrozekma.com}
\homepage{cv.mrozekma.com}
\github{mrozekma}
\gitlab{mrozekma}
\stackoverflow{309308}{Michael Mrozek}

\begin{document}

\makecvheader[R]
\makecvfooter{Generated \today. Unabridged version available from \href{https://cv.mrozekma.com/}{https://cv.mrozekma.com/}}{}{\thepage}

\untagged{technical-skills,experience,education,projects,publications}{
	\centering
	\vfill
	No sections requested\ldots
	\vfill
}

\iffalse
\cvsection{Tags}
\begin{cvparagraph}
\begin{itemize}[leftmargin=2ex, nosep, noitemsep]
\item background: \iftagged{background}{yes}{no}
\item technical-skills: \iftagged{technical-skills}{yes}{no}
	\begin{itemize}[leftmargin=2ex, nosep, noitemsep]
	\item languages-and-frameworks: \iftagged{languages-and-frameworks}{yes}{no}
	\item tools: \iftagged{tools}{yes}{no}
	\end{itemize}
\item experience: \iftagged{experience}{yes}{no}
	\begin{itemize}[leftmargin=2ex, nosep, noitemsep]
	\item mercury: \iftagged{mercury}{yes}{no}
	\item microsemi: \iftagged{microsemi}{yes}{no}
	\item arxan-defense-systems: \iftagged{arxan-defense-systems}{yes}{no}
	\item arxan: \iftagged{arxan}{yes}{no}
	\item rose-hulman: \iftagged{rose-hulman}{yes}{no}
	\item ventures: \iftagged{ventures}{yes}{no}
	\item perry-schools: \iftagged{perry-schools}{yes}{no}
	\item perritech: \iftagged{perritech}{yes}{no}
	\end{itemize}
\item education: \iftagged{education}{yes}{no}
	\begin{itemize}[leftmargin=2ex, nosep, noitemsep]
	\item masters: \iftagged{masters}{yes}{no}
	\item bachelors: \iftagged{bachelors}{yes}{no}
	\item high-school: \iftagged{high-school}{yes}{no}
	\end{itemize}
\item projects: \iftagged{projects}{yes}{no}
	\begin{itemize}[leftmargin=2ex, nosep, noitemsep]
	\item serial-bridge: \iftagged{serial-bridge}{yes}{no}
	\item chalk: \iftagged{chalk}{yes}{no}
	\item point-vote: \iftagged{point-vote}{yes}{no}
	\item ninja: \iftagged{ninja}{yes}{no}
	\item sprint: \iftagged{sprint}{yes}{no}
	\item noisebot: \iftagged{noisebot}{yes}{no}
	\item spades: \iftagged{spades}{yes}{no}
	\item got: \iftagged{got}{yes}{no}
	\item lync-helper: \iftagged{lync-helper}{yes}{no}
	\item gir: \iftagged{gir}{yes}{no}
	\item woop: \iftagged{woop}{yes}{no}
	\end{itemize}
\item publications: \iftagged{publications}{yes}{no}
	\begin{itemize}[leftmargin=2ex, nosep, noitemsep]
	\item subverting-the-fundamentals-sequence: \iftagged{subverting-the-fundamentals-sequence}{yes}{no}
	\end{itemize}
\end{itemize}
\end{cvparagraph}
\fi

\begin{taggedblock}{technical-skills}
\cvsection{Technical Skills}

\newcounter{numSkills}
\tagged{languages-and-frameworks}{\stepcounter{numSkills}}
\tagged{tools}{\stepcounter{numSkills}}

\newcommand{\skill}[2]{
	% Some tweaks are made in two-column mode to get the content to fit better
	\ifnum\thenumSkills=2
	\progressbar[width=3em]{#1}~~~\parbox[t]{.43\textwidth}{#2}\\
	\else
	\progressbar{#1}~~~#2\\
	\fi
}

\begin{paracol}{\thenumSkills}
\tagged{languages-and-frameworks}{
	\cvsubsection{Languages and Frameworks}

	\begin{cvparagraph}
	\skill{.9}{Extensive experience with C, Java, Python, Typescript, and scripting.}
	\skill{.6}{Experience with C++, Rust, and PHP.}
	\skill{.4}{Experience with SQL and miscellaneous databases, including MySQL, MariaDB, PostgreSQL, and Redis.\vspace{1mm}}
	% For some reason the space between these lines is a little cramped, so I added a tiny manual vspace after the previous skill
	\skill{.8}{Personal experience with Vue, Vite, Webpack, Yarn, and other modern web development tooling.}
	\end{cvparagraph}
}
\switchcolumn
\tagged{tools}{
	\cvsubsection{Tools}

	\begin{cvparagraph}
	\skill{.9}{Extensive experience with Linux and Windows.}
	\skill{.9}{Extensive experience with Git. Some experience with Subversion.}
	\skill{.7}{Experience with VSCode, Emacs, PyCharm, and other editors/IDEs.}
	\skill{.6}{Experience with the Atlassian toolset, including Jira, Bitbucket, and Confluence.\vspace{1mm}}
	% For some reason the space between these lines is a little cramped, so I added a tiny manual vspace after the previous skill
	\skill{.5}{Personal experience with Firebase and cloud hosting.}
	% \skill{.7}{Experience with jQuery.}
	% \skill{.5}{Experience creating GUIs in JavaFX and Qt.}
	\end{cvparagraph}
}
\end{paracol}

% Way too much space after the previous section
% \ifnum\thenumSkills=2
% \vspace{-30pt}
% \else
\vspace{-20pt}
% \fi

\end{taggedblock}
\begin{taggedblock}{experience}

\cvsection{Experience}

% \work{Tag}{Company}{Title}{When}{Content}
\newcommand{\work}[5]{
	\tagged{#1}{\cventry{#3}{#2}{#4}{}{
		\begin{cvitems}
		#5
		\end{cvitems}
	}}
}
\begin{cventries}
\work{mercury}{Mercury}{Senior Principal Software Engineer}{2016 - Present}{
	\item Designed and rolled out an automated build and test system for a number of projects.
	\item Handled the office migration to Atlassian tools, including moving all projects from Subversion to Bitbucket.
		\begin{itemize}[leftmargin=3ex, nosep, noitemsep]
		\item Implemented a tool to manage cross-repository dependencies in Bitbucket, and an addon to track additional project information within Bitbucket.
		\end{itemize}
	\item Fully automated a recurring task needed by a customer, so it can now be done on demand in about 90 minutes.
	\item Fixed and redesigned large portions of an embedded system of interconnected Microblaze cores used by many customers.
	\item Responsible for assisting newer engineers with understanding tasks, solving problems, debugging issues, and generally answering questions about the system architecture and design.
	\item Responsible for approving pull requests before they can be merged and suggesting fixes or improvements.
}%

\work{microsemi}{Microsemi}{Software Developer}{2010 - 2016}{
	\item Implemented and improved miscellaneous whitebox cryptography algorithms, including RSA, AES, and elliptic curve DSA.
	\item Developed an OpenSSL engine wrapping our whitebox cryptography implementations so they could be used by any application utilizing OpenSSL.
	\item Implemented the security model and key management policy for an Android password manager.
	\item Designed and implemented multiple higher level protocols wrapping existing whitebox cryptography to satisfy customer needs.
	\item Red- and blue-teamed several devices, finding a number of vulnerabilities. Wrote large portions of the final reports detailing the vulnerabilities, their severity, and potential mitigations and fixes.
}%

\work{arxan-defense-systems}{Arxan Defense Systems}{Software Security Analyst}{2009 - 2010}{
	\item Implemented many features in CodeSEAL, a product that analyzes executables and modifies them to protect against reverse engineering or modification.
	\item Developed WhiteboxRSA, an RSA implementation that hides the key to frustrate recovery even when the implementation is under an attacker's control.
}%

\work{arxan}{Arxan}{Intern}{2008 - 2009}{
	\item Engineering intern on a team that developed TransformIT, a white box cryptography solution.
	\item Subsequently accepted an opportunity to continue work remotely during the school year before transitioning to full-time employment after graduation.
}%

\work{ventures}{Ventures}{Team Lead}{2007 - 2008}{
	\item Head of a team of students that specified, designed, developed, and launched the web portal for a promising online startup company.
	\item Handled maintenance of development servers and software.
	\item Maintained source code repository and bug tracking system.
	\item Reviewed code produced by other team members.
}%

\work{rose-hulman}{Rose-Hulman}{Teaching Assistant ~\cdotp~ CSSE Newsletter Editor}{2005 - 2008}{
	\item Assisted students in many introductory programming classes with fundamental object-oriented techniques, UML, project planning, and completing team-based goals.
	\item Responsible for all phases of the CSSE newsletter production, from writing to publishing.
}%

\work{perry-schools}{Perry Schools}{Website Designer ~\cdotp~ Technician ~\cdotp~ Database Administrator}{2001 - 2007}{
	\item Project coordinator that designed and implemented a completely new look and feel for the school district's 5000+ page website, including a custom CMS and DB design.
	\item Converted all existing webpages to a new format and implemented a new, more efficient interface.
	\item Designed, implemented, and maintained the school's first online course registration system.
}%

\work{perritech}{Perritech}{President ~\cdotp~ Technician}{2001 - 2005}{
	\item Recruited in 2001 as a computer technician, responsible for the 2000+ computer network.
	\item Subcontracted to various local businesses for programming, website design, networking and system administration.
	\item Promoted to President in 2005, managing four directors and a number of technicians.
	\item Spearheaded a presentation at the Ohio State SchoolNet Convention illustrating the benefits of a student-run technology company fully executed within a school district. Subsequently invited to Seoul, Korea for the 6th Global Forum to make a detailed presentation on how to duplicate our success throughout many school districts.
}%
\end{cventries}
\vspace{-5pt}
\end{taggedblock}
\begin{taggedblock}{education}

\cvsection{Education}
% \school{Tag}{School}{Degree}{Location}{When}{Content}
\newcommand{\school}[6]{
	\tagged{#1}{
		\cventry{#3}{#2}{#4}{#5}{
			% At this point I don't think my GPA warrants space on my resume, so commenting out the content here
			% #6
		}
		% Since there's no content, can save some space
		\vspace{-10pt}
	}
}

\begin{cventries}
\school{masters}{Purdue University}{MS Computer Science}{West Lafayette, IN}{May 2015}{
	Cumulative GPA: 3.67
}%

\school{bachelors}{Rose-Hulman Institute of Technology}{BS Computer Science, Software Engineering (Cum Laude)}{Terre Haute, IN}{May 2009}{
	\makebox[2in][l]{Cumulative GPA: 3.48} Upperclass GPA: 3.70
}%

\school{high-school}{Perry High School}{Graduate with Honors}{Perry, OH}{May 2005}{
	\makebox[2in][l]{AP Computer Science -- 5} National AP Scholar with Distinction
}%
\end{cventries}
\vspace{-5pt}

\end{taggedblock}
\begin{taggedblock}{projects}

\cvsection{Projects}

\newcounter{numProjects}
\tagged{serial-bridge}{\stepcounter{numProjects}}
\tagged{chalk}{\stepcounter{numProjects}}
\tagged{point-vote}{\stepcounter{numProjects}}
\tagged{ninja}{\stepcounter{numProjects}}
\tagged{sprint}{\stepcounter{numProjects}}
\tagged{noisebot}{\stepcounter{numProjects}}
\tagged{spades}{\stepcounter{numProjects}}
\tagged{got}{\stepcounter{numProjects}}
\tagged{lync-helper}{\stepcounter{numProjects}}
\tagged{gir}{\stepcounter{numProjects}}
\tagged{woop}{\stepcounter{numProjects}}

% Use up to three columns for the projects, but only if there are enough projects to cover them
\newcounter{numProjectCols}
\ifnum\thenumProjects>3
\setcounter{numProjectCols}{3}
\else
\setcounter{numProjectCols}{\thenumProjects}
\fi

\newcounter{projectCurCol}
\globalcounter{projectCurCol}
% \project{Tag}{Name}{Repo URL}{Content}
\newcommand{\project}[4]{
	\tagged{#1}{
		\cvsubsection{#2} \href{#3}{\faGithubSquare}

		\begin{cvparagraph}
		#4
		\end{cvparagraph}
		\vspace{-2mm}

		\stepcounter{projectCurCol}
		\ifnum\theprojectCurCol=\thenumProjectCols
		\setcounter{projectCurCol}{0}
		% This makes every entry in the current column (that we're about to leave) the same height
		\switchcolumn*
		\else
		\switchcolumn
		\fi
	}
}

\begin{paracol}{\thenumProjectCols}
\project{serial-bridge}{Serial Bridge}{https://github.com/mrozekma/serial-bridge}{
	A web-based remote serial port interface heavily utilized at Mercury to interact with hardware UARTs.
	\vspace{-3mm}
}

\project{chalk}{Chalk}{https://github.com/mrozekma/chalk}{
	A coding interview site with a collaborative live code editor, question library, and compiler.
}

\project{point-vote}{Point Vote}{https://github.com/mrozekma/point-vote}{
	A website for hosting live scrum planning poker sessions.
}

\project{ninja}{Ninja}{https://github.com/mrozekma/ninja}{
	A website for performing calculations via a graphical data flow.
}

\project{sprint}{Sprint}{https://github.com/mrozekma/sprint}{
	A web-based Scrum tracking tool tailor-made for our project needs at Microsemi.
}

\project{noisebot}{Noisebot}{https://github.com/mrozekma/noisebot}{
	An IRC/Slack chat bot that supports reloadable modules and easy modification by end-users.
}

\project{spades}{Spades}{https://github.com/mrozekma/spades}{
	A web-based view of an IRC-hosted Spades card game. Offers a convenient view of the current game state, previous hands, and even past games.
}

\project{got}{Got}{https://github.com/mrozekma/got}{
	A command-line application to automatically download and track Git repositories. Intended to faciliate dependency resolution.
}

\project{lync-helper}{Lync Helper}{https://gitlab.com/mrozekma/lynchelper}{
	A GUI for sharing code over an office chat network. Supports syntax highlighting. Also logs conversations to text files on disk.
}

\project{gir}{Gir}{https://github.com/mrozekma/gir}{
	An ncurses editors for git interactive rebase files, which are normally edited in a regular text editor.
}

\project{woop}{Woop}{}{
	A webkit-based web browser with vim-style keybindings and modes.
}
\end{paracol}
\vspace{-5pt} % Way too much space after the previous section

\end{taggedblock}
\begin{taggedblock}{publications}
\vspace{5mm}
\cvsection{Publications}

\tagged{subverting-the-fundamentals-sequence}{
	\begin{cvparagraph}
	Clifton, C., Kaczmarczyk, L.C., Mrozek, M. ``Subverting the Fundamentals Sequence: Using Version Control to Enhance Course Management''.\\
	Presented at the 38\textsuperscript{th} Annual Technical Symposium on Computer Science Education (SIGCSE 2007).
	\end{cvparagraph}
}
\end{taggedblock}

\end{document}
